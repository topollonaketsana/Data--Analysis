\documentclass{beamer}
\usetheme{Warsaw}
\usepackage{amsmath}
\usepackage{braket}
\usepackage{caption}
\usepackage{url} 
\newcommand{\Lagr}{\mathcal{L}}
\usepackage{tikz}
\usetikzlibrary{shapes.geometric, arrows.meta, positioning}

\definecolor{myblue}{rgb}{0.0, 0.28, 0.67}
\usecolortheme[named=myblue]{structure} 

\title{Support Vector Machines}
\author{} % remove your name from footer
\institute{Department of Physics and Astronomy}
\date{05 October 2025}



% Custom footline: title on left, department on right
\setbeamertemplate{footline}{
  \leavevmode%
  \hbox{%
  \begin{beamercolorbox}[wd=.5\paperwidth,ht=2.5ex,dp=1ex,left]{author in head/foot}%
    \hspace*{2ex}\insertshorttitle
  \end{beamercolorbox}%
  \begin{beamercolorbox}[wd=.5\paperwidth,ht=2.5ex,dp=1ex,right]{author in head/foot}%
    \insertshortinstitute\hspace*{2ex}
  \end{beamercolorbox}}%
  \vskip0pt%
}


\begin{document}

% Title slide
\begin{frame}
\titlepage
\vspace{1em}
\centering
\small 
\end{frame}



\begin{frame}[t]{Introduction: SVM}\vspace{9.7mm}
\begin{itemize}
\item Support Vector Machines (SVM) is a type of supervised Machine Learning algorithm used for classification and regression
\vspace{0.5cm}
\item It constructs the optimal hyperplane in high dimensional feature space to separate data points of different classes
\vspace{0.5cm}
\item The algorith maximize the margin - distance between the decision boundary (Bernhard Boser, 1992)
\vspace{0.5cm}


\end{itemize}
\end{frame}

\begin{frame}[t]{How does SVM works?}\vspace{0.4cm}
%\large{}
%\begin{itemize}
%\item  
%\vspace{0.1cm}
%\item 
%\vspace{0.1cm}
%\item
%\end{itemize}
\vspace{0.4cm}
\centering  % center everything on the slide

\begin{figure}
    \includegraphics[width=0.7\textwidth]{plane1.png}
    \caption{1: Hyperplane (line separating two classes) - staight line in $\mathbb{R}^2$ and plane in $\mathbb{R}^3$}
\end{figure}


%\begin{figure}
%    \includegraphics[width=0.7\textwidth]{}
%    \caption{1:}
%\end{figure}





\end{frame}







\begin{frame}{How SVM works}
\begin{center}
\begin{minipage}{0.5\textwidth} % Left: figure
    \centering
    \includegraphics[width=\textwidth]{Screenshot 2025-10-15 014203.png}
    \captionof{figure}{3: SVM Illustration}
\end{minipage}%
\hfill
\begin{minipage}{0.3\textwidth} % Right: equations
    \small % makes the equations slightly smaller, but still readable
    \[
    H_1: \; \; w^TX + b_0 = 1
    \]
    \[
    H_0: \; \; w^TX + b_0 =  0
    \]
    \[
    H_2: \; \;w^TX + b_0 = -1
    \]
\end{minipage}
\end{center}

\[
\text{SVM: } \text{maximize margin } \; d\; = \frac{2}{\|w\|} \text{ subject to } y_i(w \cdot x_i + b) \ge 1
\]
% You can add more equations or text here

\begin{itemize}
\vspace{0.1cm}
\item To maximize $d,$ we must minimize the square of $w$
\end{itemize}


\end{frame}


\begin{frame}{The SVM training workflow}



\centering
\tikzstyle{box} = [rectangle, rounded corners, minimum width=6cm, 
                   minimum height=1cm, text centered, draw=blue!70, 
                   fill=blue!10, font=\small, thick]
\tikzstyle{arrow} = [thick, ->, >=Stealth]

\begin{tikzpicture}[node distance=1.5cm]

% Nodes (stacked vertically)
\node (step1) [box] {Step 1: Input data points ${x_1, x_2, x_3, \ldots, x_n}$};
\node (step2) [box, below of=step1] {Step 2: Find possible separating planes};
\node (step3) [box, below of=step2] {Step 3: Choose the maximum-margin hyperplane};
\node (step4) [box, below of=step3] {Step 4: Use support vectors to define the boundary};

% Arrows
\draw [arrow] (step1) -- (step2);
\draw [arrow] (step2) -- (step3);
\draw [arrow] (step3) -- (step4);

\end{tikzpicture}






\end{frame}







\begin{frame}{The maximum margin}
\begin{block}{Quadratic programming problem}

\[\textbf{f:} \; \; min \;\frac{1}{2} ||w||^2 \; \; s.t \;\; \textbf{g:} \; y_i(w^T \cdot x_i + b_0) - 1 = 0
\]
\end{block}
\vspace{0.3cm}
\begin{itemize}
\item We solve the above using Lagrangian multiplier method, \displaystyle \Lagr
\end{itemize}

\begin{equation}
\displaystyle \Lagr (w, b_0,\alpha)  \; =\; \frac{1}{2}||w||^2 \; + \;  \sum_i^n \alpha_i  \;[\;y_i  *(w^T x_i  +b_0 )-1\;]   
\end{equation}

\hspace{0.3cm} Where, n - number of training data points, $w_i$ - weights and  $\alpha_i$ - Lagrange multipliers
\vspace{0.1cm}
\begin{itemize}
\item To solve the $\Lagr$ we set the following conditions
\end{itemize}
\end{frame}




\begin{frame}[t]{Maximum margin}
\vspace{0.2cm}
\begin{itemize}
\item To solve the $\Lagr$ we set the following conditions
\end{itemize}

\begin{equation}
\frac{\partial L}{\partial w} = 0, \; \; \; \frac{\partial L}{\partial \alpha} = 0 , \; \; \; \frac{\partial L}{\partial b_0} = 0
\end{equation}

\begin{itemize}
\item After deriving and substituting back to $\Lagr$, we get dual optimazation problem
\end{itemize}

\[
L_d = -\frac{1}{2} \sum_i \sum_k \alpha_i \alpha_k y_i y_k (x_i)^T (x_k) + \sum_i \alpha_i
\]

such that, 

\end{frame}

\begin{frame}[t]{Example from Literature (Benhard E. Boser, 1992)}
\vspace{0.4cm}


\centering
\includegraphics[width=\textwidth]{image.png}



\begin{itemize}
\item Spport vectors (training data) ranked by alpha $\alpha_i$. Note, both rows from the same class
\end{itemize}


\end{frame}






















\begin{frame}[t]{References}\vspace{1.5cm}

\begin{itemize}
    \item Boser, B. E., Guyon, I. M., & Vapnik, V. (1992). A Training Algorithm for Optimal Margin Classifiers. \url {https://dl.acm.org/doi/pdf/10.1145/130385.130401}
    \vspace{0.2cm}

    \item 

    \vspace{0.2cm}
    
    \item 
    
    \item 
\end{itemize}
\end{frame}

\begin{frame}[plain]
\centering
\vspace{2cm}
{\Huge \textbf{Thank You!}} \\[1cm]
{\Large Questions?}
\end{frame}





\end{document}